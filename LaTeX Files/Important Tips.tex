\documentclass[a4paper,10pt]{article}
\usepackage[utf8]{inputenc}
\usepackage{amsmath}
\usepackage{amssymb}
\usepackage{gensymb}
\usepackage{chemfig}
\usepackage{wrapfig}
\usepackage[margin = 0.7in]{geometry}
\usepackage{graphicx}
\usepackage{float}
\usepackage{ragged2e}
\usepackage[document]{ragged2e}
\usepackage[table,xcdraw]{xcolor}
\usepackage[normalem]{ulem}
\usepackage{varioref}
\usepackage{hyperref}
\usepackage[nameinlink,noabbrev]{cleveref}
\useunder{\uline}{\ul}{}
\usepackage{multirow}
\usepackage{array}
\usepackage{colortbl}
\usepackage{tabularray}
\usepackage[utf8]{inputenc}
\renewcommand*\thesection{\arabic{section}.0}
\renewcommand*\thesubsection{\arabic{section}.\arabic{subsection}0}
\title{Important Tips for EE}
\date{}
\begin{document}
\maketitle
\section{Acknowledging ideas and works of another person - minimum requirements}
\begin{itemize}
    \item When writting students must clearly distinguish (in the body of the text) between their words and those of others by the use of quotation marks (or other method like indendtation) followed by an appropriate citation that denotes an entry in the bibliography..
    \item Regardless of the reference style adopted by the school for a given subject, it is expected that the minimum information given includes :
        \begin{enumerate}
            \item name of auther
            \item date of publication
            \item title of source
            \item page number as applicable 
            \item date of access
            \item URL
        \end{enumerate}
\end{itemize}
\section{Assesment Objectives}
\begin{enumerate}
    \item \textbf{Knowledge and Understanding}
        \begin{itemize}
            \item To demonstrate knowledge and understanding of the topic chosen and the research question posed.
            \item To demonstrate knowledge and understanding of subject specific terminology and/or concepts
            \item To demonstrate knowledge and understanding of relevant and/or appropriate research sources and/or methods used to gather information
        \end{itemize}
    \item \textbf{Application and Analysis}
        \begin{itemize}
            \item To select and apply research that is relevant and appropriate to the research question.
            \item To analyse the research effectively and focus on the research question
        \end{itemize}
    \item \textbf{Synthesis and evaluation}
        \begin{itemize}
            \item To be able to discuss the research in terms of a clear and coherent reasoned argument in relation to the research question.
            \item To be able to critically evaluate the arguments present in the essay
            \item To be able to reflect on and evaluate the research process. 
        \end{itemize}
    \item \textbf{A variety of (research) skills}
        \begin{itemize}
            \item To be able to present information in an appropriate academic format
            \item To understand and demonstrate academic integrity.
        \end{itemize}
\end{enumerate}
\section{Reflection}
\noindent Reflection in the EE is a critical evaluation of the decision making process. It demonstrates the evolution and discovery of conceptual understandings as they relate to the research question and sources.

\begin{itemize}
    \item Demonstrates rationale for decisions made and the skills and understandings developed, as well as the authenticity and intellectual initiative of the student voice.
    \item Show evidence of intellectual growth, critical and personal development, intellectual initiative and creativity.
    \item Refelction demonstrates the skills that have been learned, which include :
        \begin{itemize}
            \item critical thinking
            \item decision making
            \item general research
            \item planning
            \item referencing and citations
            \item Specific research methodology
            \item Time management
        \end{itemize}
    \item Explicitly assessed under assessment criterion E (Engagement).    
\end{itemize}

\section{Academic Honesty}
The essay must provide the reader with \textbf{precise} sources of quotations, ideas and points of view through accurate citations, which may be in-text or footnotes, and full references listed in the bibliography, which must meet the minimum requirements. 
\subsection{Bibliography}
\begin{itemize}
    \item Alphabetical list
    \item Sources that are not cited in the body of the essay but were important in informing the approach taken should be cited in the introduction or in an acknowledgement. 
    \item The bibliography \textbf{must} list only the sources cited.
\end{itemize}
\subsection{Citations}
Citation is a shorthand method of making reference in the body of an essay, either as an in-text citation or footnote. This must then be linked to the full reference the the end of essay in the bibliography. \\
There should be consistency of method when citing sources.

\section{Assessment Criteria}

\subsection{Criterion A: Focus and Method}
This criterion focuses on the topic, the research question and methodology. It assesses the explanation of the focus of the research (includes topic and research question), how the research will be undertaken, and how the focus is maintained throughout the essay.\\\\

\textbf{Max Marks : 5-6}
\begin{itemize}
    \item \textbf{The topic is communicated accurately and effectively}
        \begin{itemize}
            \item Identification and explanation of the research topic is effectively communicated; the purpose and focus of the research is clear and appropriate.
        \end{itemize}
    \item \textbf{Research question is clearly stated and focused}
        \begin{itemize}
            \item The research question is clear and addresses an issue of research that is appropriately connected to the discussion in the essay
        \end{itemize}
    \item \textbf{Methodology of research is complete}
        \begin{itemize}
            \item An appropriate range of relevant sources and/or methods has been selected in relation to the topic and research question.
            \item There is evidence of effective and informed selection of sources and/or methods.
        \end{itemize}    
\end{itemize}

\subsection{Criterion B: Knowledge and Understanding}
\noindent Assesses the extent to which the research related to the subject area/discipline used to explore the research question, and additionally the way in which this knowledge and understanding is demonstrated through the use of appropriate terminology and concepts.  \\\\

\textbf{Max Marks 5-6}
\begin{itemize}
    \item \textbf{Knowledge and understanding is excellent}
        \begin{itemize}
            \item The application of source materials is clearly relevant and appropriate to the research question
            \item Knowledge of the topic is clear and coherent and sources are used effectively with understanding
        \end{itemize}
    \item \textbf{Use of terminology and concepts is good}
        \begin{itemize}
            \item The use of subject-specific terminology and concepts is accurate and consistent, demonstrating effective knowledge and understanding.
        \end{itemize}    
\end{itemize}

\subsection{Criterion C: Critical Thinking}
This criterion assesses the extent to which critical thinking skills have been used to analyse and evaluate the research undertaken.
\textbf{Max Marks : 10-12}
    \begin{itemize}
        \item \textbf{The research is excellent}
            \begin{itemize}
                \item The research is appropriate to the research question and its application to support the argument is consistently relevant.
            \end{itemize}
        \item \textbf{Analysis is excellent}
            \begin{itemize}
                \item The research is analysed effectively and clearly focused on the research question; the inclusion of less relevant research does not significantly detract from the quality of the overall analysis.
                \item Conclusions to individual points of analysis are effectively supported by evidence. 
            \end{itemize}    
        \item \textbf{Discussion of evaluation is excellent}  
            \begin{itemize}
                \item An effective and focused reasoned argument is developed from the research with a conclusion reflective of the evidence presented.
                \item This reasoned argument is well structured and coherent; any minor inconsistencies do not hinder the strength of the overall argument or the final or summatve conclusion.
                \item The research has been critically evaluated.
            \end{itemize}  
    \end{itemize}

\subsection{Criterion D: Presentation}
The extent to which the presentation follows the standard format expected for academic writing and the extent to which this aids effective communication.\\\\

\textbf{Max Marks 3-4}
\begin{itemize}
    \item \textbf{Presentation is good}
        \begin{itemize}
            \item The structure of the essay clearly is appropriate in terms of the expected conversations for the topic, the argument and subject in which the essay is registered.
            \item Layout considerations are present and applied correctly
            \item The structure and layout support reading, understanding and evaluation of extended essay.
        \end{itemize}
\end{itemize}

\subsection{Criterion E: Engagement}
Assesses the student's engagement with their research focus and the research prcess. It will be applied by the examiner at the end of the assessment of the essay, and is based solely on the candidate's reflections as detailed on the RPPF, with the supervisory comments and extended essay itself as context. Only the first \textbf{500 words} are assesable. \\\\

\textbf{Max Marks: 5-6}
\begin{itemize}
    \item \textbf{Egagement is excellent}
    \begin{itemize}
        \item Reflections on decision making and planning are evaluative and include reference to the student's capacity to consider actions and ideas in responses to challenges experienced in the research process. 
        \item These reflections communicate a high degree of intellectual and personnel engagement with the research focus and process of research, demonstrating authenticity, intellectual initiative and/or creative approach in the student voice. 
    \end{itemize}
\end{itemize}

\section{Interpreting Assessment Criteria : Subject specific - Mathematics}

\subsection{Criterion A: Focus and Method}
\begin{itemize}
    \item \textbf{Strands : Topic, Research question, Methodology}
    \item Title of the essay can by itself clearly describe the topic and/or aim. It \textbf{must not be too long} and necessary clarification of it, together with a \textbf{clear indication} of the mathematical areas and the techniques, should be provided \textbf{early in the essay}.
    \item Focus and purpose of the essay must be made clear to the reader and appropriately related to the knowledge and understanding in context. \textbf{This is clearly demonstrated when the research question indicated the mathematical techniques to be applied.}
    \item The sources consulted must be sufficient and each must contribute to the research of the essay
    \item Essay must be set out in sequential form, that is each section following on from and connected to the previous one.
\end{itemize}

\subsection{Criterion B: Knowledge and Understanding}
\begin{itemize}
    \item \textbf{Strands : Context, subject specific terminology and concepts}
    \item The essay must show clear understanding of mathematics that is relevant to the focus of the essay. 
    \item Students will \textbf{not be awarded} for attempting to exhibit a wider knowledge of mathematics that is not essential to exploring research question.
    \item Students can demonstrate their understanding by :
    \begin{itemize}
        \item giving accurate and complete explanations of subject specific terminology
        \item Making knowledgeable comments on source material
        \item using source material in a relevant and appropriate way
    \end{itemize}
    \item Students need to clearly communicate and explain their mathematics. Not just talk about it but actually do the mathematics, and \textbf{must show all steps} in mathematical reasoning to\textbf{ make it clear that they understand it. }
    \item Throughout students need to demonstrate that they fully understand what they are doing.
\end{itemize}

\subsection{Criterion C: Critical Thinking}
\begin{itemize}
    \item \textbf{Strands: Research, Analysis and Discussion and evaluation}
    \item At each opportunity in essay, students must demonstrate their abilities in:
    \begin{itemize}
        \item correct deductive reasoning and argument
        \item establishing hypothesis
        \item formulating mathematical models
    \end{itemize}
    \item Students' discussion and evaluation of their result should be concise (giving lot of information in few words).
    \item Student should prove conjectures that can readily be proved.
\end{itemize}

\subsection{Criterion D: Presentation}

\begin{itemize}
    \item \textbf{Strands: Structure and layout}
    \item Concise, elegant mathematics supported by graphs, diagrams and important proofs that do not interrupt the development of essay are encouraged.
\end{itemize}

\subsection{Criterion E: Engagement}
\begin{itemize}
    \item \textbf{Strands: Reflections on planning and progress}
    \item Provide reflections on the decision making and planning processes undertaken in completing the essay. 
    \item Students \textbf{must demonstrate} how they arrived at a topic as well as the methods and approach used. 
    \item Students may reflect on :
        \begin{itemize}
            \item the approaches and strategies they chose and their relative success.
            \item the approached to learning skills they have developed and their effect on the student as a learner
            \item how their conceptual understandings have developed or changed as a result of their research
            \item challenges they faced in their research and how they overcame these
            \item  questions that emerged as a result of their research
            \item what they would do differently if they were to undertake the research again
        \end{itemize}
    \item Students must show evidence of critical and reflective thinking that goes beyond simply describing the procedures that have been followed.
    \item Reflections \textbf{must provide} examiner with insight into student thinking, creativity and originality within the research process. 
    
\end{itemize}



\end{document}

